\documentclass[12pt, a4paper]{article}
%\documentclass[12pt,openright,oneside,a4paper,english,brazil]{abntex2}

\usepackage[brazil]{babel}
\usepackage[latin1]{inputenc}
\usepackage[T1]{fontenc}
\usepackage{amssymb}

\usepackage[hidelinks]{hyperref}
\usepackage{xcolor}
\hypersetup{
    colorlinks,
    linkcolor={blue!50!green},
    citecolor={blue!50!black},
    urlcolor={blue!80!black}
}

\author{Yuri A. Martins}
\title{Fundamentos Matemáticos da Computação II}


\begin{document}

	\begin{titlepage}
		\begin{center}
			{\large Universidade Federal do Rio Grande do Norte}\\[0.2cm]
			{\large Instituto Metr\'opole Digital}\\[0.2cm]
			{\large Bacharelado em Tecnologia da Informa\c{c}\~ao}\\[0.2cm]
			{\large Fundamentos Matem\'aticos da Computa\c{c}\~ao II}\\[5.1cm]
			{\bf \huge Estudo dirigido do conte\'udo da Unidade 2}\\[5.1cm]
		\end{center}
		{\large Aluno(a): Yuri Alessandro Martins}\\[0.7cm]
		\begin{center}
			{\large Natal/RN}\\[0.2cm]
			{\large 2016}
		\end{center}
	\end{titlepage} 

	\tableofcontents
	\clearpage

	\section{Teoria dos Conjuntos}
		\subsection{Os Axiomas de Zermelo-Frankel}
			
			\subsubsection{Extensionalidade}
				Para quaisquer conjuntos A, B:\\
				$A = B ~ \Leftrightarrow ~ (\forall x ) (x \in A ~ \Leftrightarrow ~ x \in B) $
			
			\subsubsection{Emptyset}
				Garante que existe um conjunto vazio ($\emptyset$). \\
				$ \exists x \forall y ~ y \in\!\!\!\!\!/ ~~ x $
			
			\subsubsection{Pairset}
				Para todo  e b, existe o conjunto \{a,b\}.\\
				$ \forall a \forall ~ \exists w \forall x ~ (x \epsilon w ~ \Leftrightarrow ~ x = a ~\vee~ x = b ) $

			\subsubsection{Separation}
			\label{sec:separation}
				Para cada condi\c{c}\~ao P(x),\\
				$ \forall a \exists w \forall x (x \epsilon w ~ \Leftrightarrow ~ x \epsilon a ~\wedge~ x \epsilon b )$
				\\
				
				Um problema de usar somente esses \'ultimos tr\^es axiomas \'e que s\'o somos capazes de formar
				conjuntos com cardinalidade $\leq$ 2.
					\begin{itemize}
					\item ZF4 (Separation) \'e um  \underline{axiom-scheme}. Isto \'e, possui infinitos axiomas dentro dele, j\'a que
					para cada P(x) estamos formando um novo axioma.
					\end{itemize} 	
			
			\begin{description}
			\item[-] Usando os axiomas anteriores, \'e poss\'ivel representarmos algumas coisas como conjuntos:
				\begin{itemize}
				\item (x, y) = \{ \{x\}, \{x, y\}\}	\label{teste}
				\item A $\setminus$ B = \{ x $\in$ A ~$\mid$~ x $\in\!\!\!\!\!/$ B \}
				\item $ A \cap B = \{ x \in A \mid x \in B \}$ 
				\end{itemize}
			\end{description}
			
			\subsubsection{Powerset $\wp$}
				Para cada conjunto a, existe um conjunto b, onde os elementos de b s\~ao subconjuntos de a.
				$\forall a \exists p \forall w ( x \in p ~ \Leftrightarrow ~ \forall x (x \in a \Rightarrow x \in b))$

				Esse \'e o conjunto $\wp$(a).
				
				Algumas propriedades interesantes:\\
				$\wp$($\emptyset$) = $\{ \emptyset \}$ \\
				$\wp$($\{ \emptyset \} $) = $\{ \emptyset \ , \{ \emptyset \}\}$

			\subsubsection{Unionset}
				Corresponde ao conjunto $\cup a$.\\
				$\forall a \exists u \forall x ( x \in u ~\Leftrightarrow~ (\exists e \in a) [x \in e] )$

				Ex: $\cup \emptyset = \cup \{ \emptyset \} = \emptyset$

				\begin{itemize}
				\item $a \cup b = \cup \{a,b\}$
				\item $ a \times b = \{ w \in S \mid \exists x \exists y (w = (x,y) \land x \in a \land x \in b)\}$ \\
						Onde $S = \wp(\wp(a \cup b))$
				\item sigletonset = $\{ x ~ \in ~ \wp a ~\mid~ (\exists t ~\in~ a)[x = \{t\}]\}$
				\item $\cap a = \{x ~\in~ \cup a ~\mid~ (\forall e ~\in~ a)[x ~\in~ e]\}$
				\end{itemize}

			\subsubsection{Infinity Axiom}
				$\exists I (\emptyset ~\in~ I ~\land~ \forall x (x ~\in~ I \Rightarrow \{x\} ~\in~ I ))$ or\\
				$\exists I (\emptyset ~\in~ I ~\land~ \forall x (x ~\in~ I \Rightarrow x ~\cup~ \{x\} ~\in~ I ))$\\

				Esse axioma \'e garantido pois\\
				$\{x\} \ne x $\\
				$ x \cup \{x\} \ne x $

			\subsection{Rela\c{c}\~oes, Fun\c{c}\~oes e Fun\c{c}\~oes parciais de Zermelo-Frankel}
			\subsubsection{Definindo um par ordenado}
				A opera\c{c}\~ao de par do Kuratowski:\\
				$ (x, y) = \{ \{x\}, \{x, y\}\} $, como vimos anteriormente na se\c{c}\~ao~\ref{sec:separation}

			\subsubsection{Rela\c{c}\~oes}
				
				Def: Sejam A,B conjuntos, R \'e uma rela\c{c}\~ao \underline{entre} A e B se R $\subseteq$ A $\times$ B.

				Dessa forma, 
				\begin{itemize}
				\item f(a) = b $\to$ (a,b) $\in$ f
				\end{itemize}

				\begin{itemize}
				\item[-] Uni\~ao Disjunta: $A \uplus B = (\{0,a\} \times A) \cup (\{1,b\} \times B)$				
				\end{itemize}


				Sendo R uma rela\c{c}\~ao sobre o conjunto $\mathbb{N}$ (R $\subseteq$ A $\times$ A),
				\begin{itemize}
				\item x R x : Reflexiva - `` $ =, ~\leq, ~\geq, ~\subseteq $ ''
				\item x R y $\Rightarrow $ y R x  : Sim\'etrica - `` $ = $  ''
				\item x R y $\land$ y R z $\Rightarrow$ x R z : Transitiva -  `` $ =, ~\leq, ~\geq, ~<, ~>, ~\subseteq $ ''
				\end{itemize}

			\subsubsection{Fun\c{c}\~oes e Fun\c{c}\~oes parciais}

		\subsection{Cardinais}
		\subsection{Os Axiomas de Peano e os Naturais na ZFC}
		\subsection{Teorema da Recurs\~ao}
		\subsection{String Recursion}

	\section{$\lambda$-Calculus}
		\subsection{O conjunto de $\lambda$-termos}
		\subsection{$\alpha,~ \beta~ e~ \eta~$ conversions}
		\subsection{Booleanos naturais no $\Lambda$}
		\subsection{Combinators I, K, B, S}


\end{document}