\documentclass[12pt, a4paper]{article}
%\documentclass[12pt,openright,oneside,a4paper,english,brazil]{abntex2}

\usepackage[brazil]{babel}
\usepackage[utf8]{inputenc}
\usepackage[T1]{fontenc}
\usepackage{amsmath,amssymb,latexsym}
%\usepackage{amssymb}
\usepackage{amsthm}
%\usepackage{amsmath}
\usepackage{amstext}
\usepackage{multicol}
\usepackage{qtree}
\usepackage{mathtools}
\usepackage{cite}

\usepackage[hidelinks]{hyperref}
\usepackage{xcolor}
\hypersetup{
colorlinks,
linkcolor={blue!50!green},
citecolor={blue!50!black},
urlcolor={blue!80!black}
}

\newtheorem*{theorem}{Theorem}

\author{Yuri A. Martins}
\title{Fundamentos Matemáticos da Computação II}


\begin{document}

\begin{titlepage}
\begin{center}
{\large Universidade Federal do Rio Grande do Norte}\\[0.2cm]
{\large Instituto Metrópole Digital}\\[0.2cm]
{\large Bacharelado em Tecnologia da Informação}\\[0.2cm]
{\large Fundamentos Matemáticos da Computação II}\\[5.1cm]
{\bf \huge Estudo dirigido do conteúdo da Unidade 2}\\[5.1cm]
\end{center}
{\large Autor: Yuri Alessandro Martins}\\[0.7cm]
\begin{center}
{\large Natal/RN}\\[0.2cm]
{\large \today}
\end{center}
\end{titlepage} 

\tableofcontents
\clearpage

\section{Teoria dos Conjuntos}
\subsection{Os Axiomas de Zermelo-Frankel}

\subsubsection{Extensionalidade}
\label{sec:extensionalidade}
Para quaisquer conjuntos $A, B$:\\
\begin{center}
$A = B ~ \iff ~ (\forall x ) (x \in A ~ \iff ~ x \in B) $
\end{center}
\subsubsection{Emptyset}
\label{sec:emptyset}
Garante que existe um conjunto vazio ($\emptyset$). \\
\begin{center}
$ \exists x \forall y ~ y \in\!\!\!\!\!/ ~~ x $
\end{center}
\subsubsection{Pairset}
Para todo $a$ e $b$, existe o conjunto $\{a,b\}$.\\
\begin{center}
$ \forall a \forall ~ \exists w \forall x (x \epsilon w ~ \iff ~ x = a ~\vee~ x = b ) $
\end{center}
\subsubsection{Separation}
\label{sec:separation}
Para cada condição $P(x)$,\\
\begin{center}
$ \forall a \exists w \forall x (x \epsilon w ~ \iff ~ x \epsilon a ~\wedge~ x \epsilon b )$
\end{center}

Um problema de usar somente esses últimos três axiomas é que só somos capazes de formar
conjuntos com cardinalidade $\leq$ 2.
\begin{itemize}
\item ZF4 (\ref{sec:separation}) é um  \underline{axiom-scheme}. Isto é, possui infinitos axiomas dentro dele, já que
para cada P(x) estamos formando um novo axioma.
\end{itemize} 	

\begin{description}
\item[-] Usando os axiomas anteriores, é possível representarmos algumas coisas como conjuntos:
\begin{itemize}
\item (x, y) $\triangleq$ \{ \{x\}, \{x, y\}\}	\label{teste}
\item A $\setminus$ B $\triangleq$ \{ x $\in$ A ~$\mid$~ x $\in\!\!\!\!\!/$ B \}
\item $ A \cap B \triangleq \{ x \in A \mid x \in B \} $ 
\end{itemize}
\end{description}

\subsubsection{Powerset $\wp$}
Para cada conjunto $a$, existe um conjunto $b$, onde os elementos de $b$ são subconjuntos de $a$.
\begin{center}
$\forall a \exists p \forall w ( x \in p ~ \iff ~ \forall x (x \in a \Rightarrow x \in b))$
\end{center}

Esse é o conjunto $\wp$(a).

Aqui $x \in a$ é uma abreviação de $(\forall t)[t \in x \Rightarrow t \in a]$. O Axioma da Extensionalidade (\ref{sec:extensionalidade}) implica que para cada a, apenas um conjunto b pode satisfazer a defininção do Powerset; Nós podemos chamar \textbf{Conjunto Potência} de a e denotá-lo como:\\
$\wp(a) \triangleq \{x \mid Set(x) \& x \in a \}$\\

Algumas propriedades interesantes:\\
$\wp$($\emptyset$) = $\{ \emptyset \}$ \\
$\wp$($\{ \emptyset \} $) = $\{ \emptyset \ , \{ \emptyset \}\}$\\

Exercício: Para cada conjunto $A$, existe um conjunto $B$ cujo membros são exatamente singletons dos membros de $A$:
\begin{center}
$ x \in B \iff (\exists t \in A) [x = \{t\}]$
\end{center}
\subsubsection{Unionset}
\label{sec:unionset}
Corresponde ao conjunto $\cup a$.\\
\begin{center}
$\forall a \exists u \forall x ( x \in u ~\iff~ (\exists e \in a) [x \in e] )$
\end{center}

Ex: $\cup \emptyset = \cup \{ \emptyset \} = \emptyset$

\begin{itemize}
\item $a \cup b = \cup \{a,b\}$

Usando os axiomas ZF2 (\ref{sec:emptyset}) e ZF5 (\ref{sec:unionset})\\
\begin{center}
$t \in A \cup B \iff (\exists X \in \{A,B\})[t \in X]$\\
$t \in A \cup B \iff t \in A \vee t \in B$
\end{center}
\item $ a \times b \triangleq \{ w \in S \mid \exists x \exists y (w = (x,y) \land x \in a \land x \in b)\}$ \\
Onde $S = \wp(\wp(a \cup b))$
\item sigletonset $ \triangleq \{ x ~ \in ~ \wp a ~\mid~ (\exists t ~\in~ a)[x = \{t\}]\}$
\item $\cap a \triangleq \{x ~\in~ \cup a ~\mid~ (\forall e ~\in~ a)[x ~\in~ e]\}$
\end{itemize}

\subsubsection{Infinity Axiom}
\label{sec:infinity}
$\exists I (\emptyset ~\in~ I ~\land~ \forall x (x ~\in~ I \Rightarrow \{x\} ~\in~ I ))$ ou\\
$\exists I (\emptyset ~\in~ I ~\land~ \forall x (x ~\in~ I \Rightarrow x ~\cup~ \{x\} ~\in~ I ))$\\

Esse axioma é garantido pois\\
\begin{align*}
\{x\} &\ne x \\
x \cup \{x\} &\ne x 
\end{align*}

Com ele, somos capazes de montar o seguinte conjunto infinito:\\
I = $\{ \emptyset, \{\emptyset\}, \{\{\emptyset\}\}, \cdots \}$

\subsection{Relações, Funções e Funções parciais na ZFC}
\subsubsection{Definindo um par ordenado}
A operação de par do Kuratowski:\\
$ (x, y) = \{ \{x\}, \{x, y\}\} $, como vimos anteriormente na seção~\ref{sec:separation}\\

PROOF AND MORE DETAILS...

\subsubsection{Relações}

Def: Sejam $A,B$ conjuntos, $R$ é uma relação \underline{entre} $A$ e $B$ se $R \subseteq A \times B$.

Dessa forma, 
\begin{itemize}
\item $f(a) = b \rightsquigarrow (a,b) \in f$
\end{itemize}

\begin{itemize}
\item[-] União Disjunta: $A \uplus B = (\{0,a\} \times A) \cup (\{1,b\} \times B)$				
\end{itemize}

Sendo $R$ uma relação sobre o conjunto $\mathbb{N} (R \subseteq A \times A), R$ pode ser:

\begin{align*}
x R x &: Reflexiva \rightsquigarrow `` =, ~\leq, ~\geq, ~\subseteq ''\\
x R y \Rightarrow  y R x  &: Simetrica \rightsquigarrow `` =  ''\\
x R y \land y R z \Rightarrow x R z &: Transitiva \rightsquigarrow  `` =, ~\leq, ~\geq, ~<, ~>, ~\subseteq ''
\end{align*}	

Ainda existem outras propriedas como essas, como a \underline{Antireflexiva} ou \underline{Antisimétrica}.

\subsubsection{Relações de Equivalência}
Uma relação sobre um conjunto $A$ é chamada \textbf{relação de equivalência} se ela for \underline{reflexiva},
\underline{simétrica} e \underline{transitiva}.

O conjunto de todos os elementos que são relacionados a um elemento $a$ de $A$ é chamado de \underline{classe de equivalência} de $a$. Isso implica que:
\begin{center}
$\cup [a] = A$
\end{center}

\begin{itemize}
\item $[a] \cap [b] = \emptyset quando [a] \neq[b]$\\
\end{itemize}

Uma partição de um conjunto $S$ é uma coleção de subconjuntos disjuntos não vazios de $S$. A união de todas as partições resulta, portanto, em $S$. Em outras palavras, os subconjuntos $A_i$ formam partições de $S$ se e somente se\\
\begin{align*}
A_i &\neq \emptyset\\
A_i &\cap A_j = \emptyset, quando~i \neq j\\
\cap A_i &= S
\end{align*}

Podemos definir classes de equivalência como:\\
\begin{align*}
[x/\backsim] &\triangleq \{a \in A \mid x \backsim a \}\\
[A/\backsim] &\triangleq \{c \in \wp(A) \mid \exists x ~ C = [x/\backsim] \}
\end{align*}

Seja $x,y \in A$, e $\backsim$ uma relação de equivalência no $A$:
\begin{align*}
[x/\backsim] = [y/\backsim] &\iff x \backsim y\\
[x/\backsim] = [y/\backsim] &\iff
\begin{cases} 
[x/\backsim] 	& \text{se } x \backsim y \\
\emptyset 		& \text{se não}\\
\end{cases}
\\
\cup\{[x/ \backsim] \mid x \in a \} &= A
\end{align*}

\subsubsection{Funções}

\subsection{Currying}
Dada uma função $f$ do tipo $f:(X \times X) \rightarrow Z$, então a técnica de \textbf{currying} a torna $(f): X \rightarrow (Y \rightarrow Z)$.
Isto é, currying torna um paramêtro do tipo $X$ e retorna uma função do tipo $Y \rightarrow Z$.\\


\noindent Achar um $\phi ( (x,y) \rightarrow A \rightarrowtail\!\!\!\!\!\rightarrow (x \rightarrow (y \rightarrow A)))$\\
$\phi(F) = G$, onde $G$ é definida pela,\\
$G(x)(y) = g$, ode $g$ é definida pela,\\
$g(y), f(x,y)$

\subsection{Cardinais}
Seja $A$ um conjunto. O que é $|A|$?
\begin{itemize}
\item[c1.] $A =_c |A|$
\item[c2.] $A =_c B \iff |A| = |B|$
\item[c3.] para todo conjunto de conjuntos $\epsilon$,\\
$\{|x| \mid x \in \epsilon \}$ é conjunto.
\end{itemize}

Aqui estaremos definindo funções cardinais fracas (\textbf{weak}). Isso porquê seria necessário provar o c2, algo extremamanete complicado agora. Portanto, podemos o resumir como:\\

\begin{itemize}
\item[c2.] $A =_c B \iff |A| =_c |B|$
\end{itemize}

Sejam $\kappa, \lambda, \mu$ números cardinais:
\begin{itemize}
\item $\kappa + \lambda \triangleq_c \kappa \uplus \lambda$
\item $\kappa . \lambda \triangleq_c \kappa \times \lambda$
\item $\kappa^\lambda \triangleq_c (\kappa \to \lambda)$
\end{itemize}

\subsection{Os Axiomas de Peano}
\textbf{Structed set}: $(\mathbb{N}; 0; S)$, onde $0 \in \mathbb{N}$ e $S:\mathbb{N} \rightarrow \mathbb{N}$\\

\begin{align*}
0 &\in \mathbb{N}\\
S: \mathbb{N} &\rightarrow \mathbb{N}\\
S: \mathbb{N} &\rightarrowtail \mathbb{N}\\
(\forall x \in \mathbb{N})&[S_n \neq 0 ]\\
(\forall x \subseteq \mathbb{N})&[ [0 \in X \land (\forall n \in \mathbb{N}) [n \in X \Rightarrow S_n \in X ]] \Rightarrow X = \mathbb{N}]
\end{align*}

O axioma de peano 5 é o que nos permite realizar indução matemática. Observe:
\begin{align*}
\forall x \subseteq \mathbb{N}~&corresponde~a~\textbf{base}.\\
(\forall n \in \mathbb{N}) [n \in X \Rightarrow S_n \in X] ~&corresponde~ao~\textbf{passo indutivo}.\\
n \in X ~&corresponde~a~\textbf{hipotése indutiva}.
\end{align*}

\subsection{Teorema da Recursão}
\label{sec:recursion}
\begin{theorem}
Sejam: $(\mathbb{N}, 0, S)$ um sistema de naturais conjunto E.\\
$a \in E$\\
$h:E \rightarrow E$\\
Então existe $f: \mathbb{N} \rightarrow E$\\
tal que: $f(0) = a e f(S_n) = h(f(n))$.
\end{theorem}


\subsection{Os Naturais na ZFC}
Para estabelecermos os Naturais na ZFC, temos que garantir duas coisas:
\begin{itemize}
\item Existência de $\mathbb{N}$
\item Singularidade de $\mathbb{N}$
\end{itemize}

Para tal, iremos precisar do Teorema da Recursão (\ref{sec:recursion}).

\subsubsection{Existência de $\mathbb{N}$}
Seja $J$ = todos os conjuntos $X$ tal que satisfaz o ZF7 (\ref{sec:infinity})\\
$J = \{X \in \wp I \mid \emptyset \in X \land (\forall x \in X)[S \in X]\}$\\
\begin{multicols}{2}
\noindent Seja $\mathbb{N} \triangleq \cap J$\\ 
Seja $0_1 = \emptyset$\\ 
Seja $S_1 = \lambda x.\{x\}$ \footnote{Visiste \ref{sec:lambda} para $\lambda$-Calculus e entender melhor esse ponto}\\ 

\noindent Seja $\mathbb{N} \triangleq \cap J$\\
Seja $0_2 = \emptyset$\\
Seja $S_2 = \lambda x.x \cup \{x\} $ \footnote{Veja nota 1} 
\end{multicols}

Encaixando com os Axiomas de Peano:
\begin{enumerate}
\item $(\forall x \in J)[\emptyset \in X]$, então $\emptyset \in \cap J$ e $\emptyset \in \mathbb{N}$
\item Também, pela própria definição de J
\item $a \neq b \iff S_a \neq S_b$ ou $a \neq b \iff \{a\} \neq \{b\}$
\item $\forall x \{x\} \neq \emptyset$
\item Seja $X \subseteq \mathbb{N}$, tal que\\	
0 $\in$ X\\
$(\forall x \in X)[S_X \in X]$\\
Seja $n \in \mathbb{N}$\\
$\exists p : x = \{p\}$\\
--> Mesmo que $\{p\} \in \cap J = \mathbb{N}$\\
--> Mesmo que $n \in X$ e $x \geq \mathbb{N}$
\end{enumerate}

Basicamente, podemos descrever $\mathbb{N}$ de duas maneiras, agora:
\begin{multicols}{2}
\noindent 0 \hspace{4bp} $\emptyset$ = $\emptyset$\\
1 \hspace{4bp} $\{\emptyset\} = \{0\}$\\
2 \hspace{4bp} $\{\{\emptyset\}\} = \{1\}$\\
3 \hspace{4bp} $\{\{\{\emptyset\}\}\} = \{2\}\\$
$\vdots$\\
\hspace{5bp} S = $\lambda$x.$\{x\}$ \\

\noindent 0 \hspace{4bp} $\emptyset$ = $\emptyset$\\
1 \hspace{4bp} $\{\emptyset\} = \{0\}$\\
2 \hspace{4bp} $\{\emptyset, \{\emptyset\}\} = \{0,1\}$\\
3 \hspace{4bp} $\{\emptyset, \{\emptyset\}\, \{\{\emptyset\}\}\}$\\
$\vdots$\\
\hspace{5bp} S = $\lambda x.x \cup \{x\} $
\end{multicols}

\subsubsection{Singularidade de $\mathbb{N}$}
``$\mathbb{N}$ is unique up to isomorphism:''\\
$\pi$ : ($\mathbb{N}_1$; $0_1$; $S_1$) $ \rightarrowtail\!\!\!\!\!\rightarrow $ ($\mathbb{N}_1$; $0_2$; $S_2$).
\begin{itemize}
\item[]\hspace{4bp}$\pi (0_1) = 0_2$
\item[]\hspace{4bp}$\pi (S_1 n_1) = S_2 \pi (n_1)$
\end{itemize}

Se traçarmos um paralelo com o Teorema da Recursão (\ref{sec:recursion}), para tentarmos provar a singularidade de $\mathbb{N}$, podemos realizar as seguintes associações:
\begin{itemize}
\item $\mathbb{N}$ : $\mathbb{N}_1$
\item E: $\mathbb{N}_2$
\item a: $0_2$
\item h: $S_2$
\end{itemize}

\begin{proof}
$\pi : \mathbb{N}_1 \Rightarrow \mathbb{N}_2$\\
$\pi[\mathbb{N}_1] = \mathbb{N}_2$
\begin{itemize}
\item $0_2 \in \pi[\mathbb{N}_1]$\\
--> Como $\pi(0_1) \Rightarrow S_2n_2 \in \pi[\mathbb{N}_1]$
\item $n_2 \in \pi[\mathbb{N}_1]$\\
--> Suponha que $n_2 \in \pi[\mathbb{N}_1]$ --> H.I\\
--> $(\exists n_1 \in \mathbb{N}_1)[\pi(n_1) = n_2]$\\
--> $\pi(S_1n_1) = S_2(\pi(n_2))$ que $ = S_2n_2$\\
\end{itemize}

MAIS COISA AQUI DEPOIS...[?]

\end{proof}

\subsection{String Recursion}
\begin{itemize}
\item Dado [ ] $\in$ [$\mathbb{N}$]
\item Se $n \in \mathbb{N}$, e $L \in [\mathbb{N}]$, então $(n:L) \in [\mathbb{N}]$
\end{itemize}

Exemplo: 2:3:4:[~~] = [2,3,4]\\

Alguns exemplos de funções recursivas que podemos definir utilizando String Recursion:

\begin{flalign*}
iszero : [\mathbb{N}] \rightarrow \mathbb{B}\\
iszero~0 &= true&\\
iszero~S_n &= false &
\end{flalign*}

\begin{flalign*}
empty : [\mathbb{N}] \rightarrow \mathbb{B}\\
empty~[~~] &= true&\\
empty~(x:x_s) &= false&
\end{flalign*}

\begin{flalign*}
++ : [\mathbb{N}] \rightarrow [\mathbb{N}] \rightarrow [\mathbb{N}]\\
[~~] ~ ++ ~ y_s &= y_s&\\
(x:x_s) ~ ++ ~ y_s &= x:(x_s ++ y_s)&
\end{flalign*}

\begin{flalign*}
Ex: [1,2] ++ [6,7,8,9] &= 1:2:[~~] ++ [6,7,8,9]&\\
&= 1:(2:[~~] ++ [6,7,8,9])&\\
&= 1:(2:([~~] ++ [6,7,8,9]))&\\
&= 1:2:[6,7,8,9]&\\
&= [1,2,6,7,8,9]&
\end{flalign*}

\begin{flalign*}
reverse : [\mathbb{N}] \rightarrow [\mathbb{N}]\\
reverse [~~] &= [~~]&\\
reverse [x] &= [x]&\\
revese (x:x_s) &= reverse xs ++ [x]&
\end{flalign*}

\begin{flalign*}
\sqsubseteq : [\mathbb{N}] \rightarrow [\mathbb{N}] \rightarrow \mathbb{B}\\
(x:x_s) \sqsubseteq [~~] &= false&\\
'[~~] \sqsubseteq y_s &= true&\\
(x:x_s) \sqsubseteq (y:y_s) &= (x = y) \land xs \sqsubseteq ys&
\end{flalign*}

\begin{flalign*}
Ex: [2,3,4,5] \sqsubseteq [2,3,5,7] &= (2 = 2) \land ([3,4,5] \sqsubseteq [3,5,7])&\\
&= (3 = 3) \land ([4,5] \sqsubseteq [5,7])&\\
&= (4 = 5) \land ([5] \sqsubseteq [7])&\\
&= \text{FALSE}&
\end{flalign*}

\begin{flalign*}
\in : \mathbb{N} \rightarrow [\mathbb{N}] \rightarrow \mathbb{B}&\\
n \in x [~~] &= false&\\
x \in (x:x_s) &= (n = x) \vee (n \in x_s)&
\end{flalign*}

\begin{flalign*}
find : \mathbb{N} \rightarrow [\mathbb{N}] \rightarrow \mathbb{B}\\
find n [~~] &= 0&\\
find n (n:nx) &= 0&\\
find n (x:x_s) &= 1 + find~n~x_s&
\end{flalign*}

\begin{flalign*}
sum : [\mathbb{N}] \rightarrow \mathbb{N}&\\
sum [~~] &= 0 & \\
sum (x:xs) &= x + sum xs &
\end{flalign*}

\begin{flalign*}
\oplus : [\mathbb{N}] \rightarrow [\mathbb{N}] \rightarrow [\mathbb{N}]\\
'[~~] \oplus y_s &= y_s &\\
x_s \oplus [~~] &= x_s &\\
(x:x_s) \oplus (y:y_s) &= (x+y):(x_s \oplus y_s) &
\end{flalign*}

\begin{flalign*}
circle :(\mathbb{N} \rightarrow \mathbb{N} \rightarrow \mathbb{N}) \rightarrow [\mathbb{N}] \rightarrow [\mathbb{N}] \rightarrow [\mathbb{N}]\\
circle f [~~] y_s &= [~~]\\
f xs [~~] &= [~~]\\
f (x:x_s) (y:y_s) &= [f x y]:(circle~f~x_s~y_s)
\end{flalign*}

\section{$\lambda$-Calculus}
\label{sec:lambda}
\subsection{O conjunto de $\lambda$-termos}
Sendo $\Lambda = \lambda$-termos;
\begin{flalign*}
X &\in \Lambda&\\
s,t &\in \Lambda \Rightarrow (s~t) \in \Lambda&\\
x \in var, t \in \Lambda &\Rightarrow \lambda X.t \in \Lambda&
\end{flalign*}

\subsection{Conversões $\alpha,~ \beta~ e~ \eta~$}
\subsubsection{Conversão $\alpha$}
Determina que a escolha da variável ligada, na abstração lambda, não importa (normalmente):
\begin{flalign*}
\lambda x.x &=_\alpha \lambda y.y&\\
\lambda x.\lambda x.x &=_\alpha \lambda y.\lambda x.x & && \text{Note que isso não poderá ser transformado em } \lambda y.\lambda x.y
\end{flalign*}

Primeiro, quando alfa-conversão atua em uma abstração, as únicas ocorrências de variáveis que podem ser renomeados são aqueles que são vinculados a esta mesma abstração. No segundo exemplo, portanto:

\begin{flalign*}
\lambda x.\lambda x.x &\not=_\alpha \lambda y.\lambda x.y & &&\text{Este último tem um significado diferente do original.}
\end{flalign*}

Em segundo lugar, uma conversão $\alpha$ não é possível se isto irá resultar em uma variável sendo capturada por uma abstração diferente. Por exemplo, se substituirmos $x$ com $y$ em $\lambda x.\lambda y.x$, nós obteríamos $\lambda y.\lambda y.y$, que tem um significado diferente da expressão anterior.

\subsubsection{Redução $\beta$}
Redução $\beta$ é a ideia de aplicar uma função. Por exemplo, se temos $f(x) = x*2$, para $x = 2$ aplicamos o valor a função que irá ficar como $f(2) = 2*2$. Essa é basicamente a ideia da redução $\beta$.
\begin{flalign*}
(\lambda x.x*2)~2 &=_\beta 2*2 &
\end{flalign*}
\subsubsection{Redução $\eta$}
Eta-conversão expressa a ideia de extensionalidade, que neste contexto é que duas funções são as mesmas se e somente se eles dão o mesmo resultado para todos os argumentos.\footnote{Sujeito a severas mudanças no futuro. Visite \ref{sec:colaboracao} para saber mais sobre.}

\subsection{Booleanos naturais no $\Lambda$}
\begin{align*}
\lambda y.x = \lambda x.(\lambda y.x) &\coloneqq true \coloneqq fst\\
\lambda x.y = \lambda x.(\lambda y.y) &\coloneqq false \coloneqq snd
\end{align*}
\subsection{Combinators I, K, B, S}
Combinadores\footnote{A melhorar bruscamente}

\begin{flalign*}
I &= \lambda x.x &\\
B &= \lambda x \lambda y . \lambda z . x (yz) & &&\text{``Composition''}\\
S &= \lambda x \lambda y . \lambda z. \lambda xz(yz) &
\end{flalign*}

\section{Política de Colaboração}
\label{sec:colaboracao}
Você é capaz de alterar o conteúdo desse documento, para corrigir erros, melhorar suas explicações ou dar dicas/exemplos adicionais. Esse foi o objetivo desde começo.

\href{https://github.com/YuriAlessandro/BTI-Documents/blob/master/FMC2/fmc2_un2.tex}{Visita a página remota do documento} para obter sua versão mais atualizada e/ou colaborar também.

Como base foram utilizados os livros ``Notes on Set Theory''\cite{moschovakis2006notes} e ``Classic Set Theory''\cite{goldrei1996classic}. Caso você queira continuar usando-os como base para esse documento, sinta-se a vontade.

\clearpage
\bibliographystyle{plain}
\bibliography{fmc2_un2_refs}

\end{document}
