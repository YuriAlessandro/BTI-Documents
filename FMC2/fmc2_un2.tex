\documentclass[12pt, a4paper]{article}
%\documentclass[12pt,openright,oneside,a4paper,english,brazil]{abntex2}

\usepackage[brazil]{babel}
\usepackage[latin1]{inputenc}
\usepackage[T1]{fontenc}
\usepackage{amssymb}
\usepackage{amsthm}
\usepackage{amsmath}
\usepackage{amstext}
\usepackage{multicol}
\usepackage{qtree}


\usepackage[hidelinks]{hyperref}
\usepackage{xcolor}
\hypersetup{
    colorlinks,
    linkcolor={blue!50!green},
    citecolor={blue!50!black},
    urlcolor={blue!80!black}
}

\author{Yuri A. Martins}
\title{Fundamentos Matemáticos da Computação II}


\begin{document}

	\begin{titlepage}
		\begin{center}
			{\large Universidade Federal do Rio Grande do Norte}\\[0.2cm]
			{\large Instituto Metr\'opole Digital}\\[0.2cm]
			{\large Bacharelado em Tecnologia da Informa\c{c}\~ao}\\[0.2cm]
			{\large Fundamentos Matem\'aticos da Computa\c{c}\~ao II}\\[5.1cm]
			{\bf \huge Estudo dirigido do conte\'udo da Unidade 2}\\[5.1cm]
		\end{center}
		{\large Aluno(a): Yuri Alessandro Martins}\\[0.7cm]
		\begin{center}
			{\large Natal/RN}\\[0.2cm]
			{\large \today}
		\end{center}
	\end{titlepage} 

	\tableofcontents
	\clearpage

	\section{Teoria dos Conjuntos}
		\subsection{Os Axiomas de Zermelo-Frankel}
			
			\subsubsection{Extensionalidade}
			\label{sec:extensionalidade}
				Para quaisquer conjuntos A, B:\\
				\begin{center}
				$A = B ~ \Leftrightarrow ~ (\forall x ) (x \in A ~ \Leftrightarrow ~ x \in B) $
				\end{center}
			\subsubsection{Emptyset}
			\label{sec:emptyset}
				Garante que existe um conjunto vazio ($\emptyset$). \\
				\begin{center}
				$ \exists x \forall y ~ y \in\!\!\!\!\!/ ~~ x $
				\end{center}
			\subsubsection{Pairset}
				Para todo a e b, existe o conjunto \{a,b\}.\\
				\begin{center}
				$ \forall a \forall ~ \exists w \forall x ~ (x \epsilon w ~ \Leftrightarrow ~ x = a ~\vee~ x = b ) $
				\end{center}
			\subsubsection{Separation}
			\label{sec:separation}
				Para cada condi\c{c}\~ao P(x),\\
				\begin{center}
				$ \forall a \exists w \forall x (x \epsilon w ~ \Leftrightarrow ~ x \epsilon a ~\wedge~ x \epsilon b )$
				\end{center}
					
				Um problema de usar somente esses \'ultimos tr\^es axiomas \'e que s\'o somos capazes de formar
				conjuntos com cardinalidade $\leq$ 2.
					\begin{itemize}
					\item ZF4 (~\ref{sec:separation}) \'e um  \underline{axiom-scheme}. Isto \'e, possui infinitos axiomas dentro dele, j\'a que
					para cada P(x) estamos formando um novo axioma.
					\end{itemize} 	
			
			\begin{description}
			\item[-] Usando os axiomas anteriores, \'e poss\'ivel representarmos algumas coisas como conjuntos:
				\begin{itemize}
				\item (x, y) $\triangleq$ \{ \{x\}, \{x, y\}\}	\label{teste}
				\item A $\setminus$ B $\triangleq$ \{ x $\in$ A ~$\mid$~ x $\in\!\!\!\!\!/$ B \}
				\item $ A \cap B \triangleq \{ x \in A \mid x \in B \} $ 
				\end{itemize}
			\end{description}
			
			\subsubsection{Powerset $\wp$}
				Para cada conjunto a, existe um conjunto b, onde os elementos de b s\~ao subconjuntos de a.
				\begin{center}
				$\forall a \exists p \forall w ( x \in p ~ \Leftrightarrow ~ \forall x (x \in a \Rightarrow x \in b))$
				\end{center}

				Esse \'e o conjunto $\wp$(a).

				Aqui $x \in a$ \'e uma abrevia\c{c}\~ao de $(\forall t)[t \in x \Rightarrow t \in a]$. O Axioma da Extensionalidade~(\ref{sec:extensionalidade}) implica que para cada a, apenas um conjunto b pode satisfazer a definin\c{c}\~ao do Powerset; N\'os podemos chamar \textbf{Conjunto Pot\^encia} de a e denot\'a-lo como:\\
				$\wp(a) \triangleq \{x \mid Set(x) \& x \in a \}$\\
				
				Algumas propriedades interesantes:\\
				$\wp$($\emptyset$) = $\{ \emptyset \}$ \\
				$\wp$($\{ \emptyset \} $) = $\{ \emptyset \ , \{ \emptyset \}\}$\\

				Exerc\'icio: Para cada conjunto A, existe um conjunto B cujo membros s\~ao exatamente singletons dos membros de A:
				\begin{center}
				$ x \in B \Leftrightarrow (\exists t \in A) [x = \{t\}]$
				\end{center}
			\subsubsection{Unionset}
			\label{sec:unionset}
				Corresponde ao conjunto $\cup a$.\\
				\begin{center}
				$\forall a \exists u \forall x ( x \in u ~\Leftrightarrow~ (\exists e \in a) [x \in e] )$
				\end{center}

				Ex: $\cup \emptyset = \cup \{ \emptyset \} = \emptyset$

				\begin{itemize}
				\item $a \cup b = \cup \{a,b\}$
				
				Usando os axiomas ZF2 (\ref{sec:emptyset}) e ZF5 (\ref{sec:unionset})\\
				\begin{center}
				$t \in A \cup B \Leftrightarrow (\exists X \in \{A,B\})[t \in X]$\\
				$t \in A \cup B \Leftrightarrow t \in A \vee t \in B$
				\end{center}
				\item $ a \times b \triangleq \{ w \in S \mid \exists x \exists y (w = (x,y) \land x \in a \land x \in b)\}$ \\
						Onde $S = \wp(\wp(a \cup b))$
				\item sigletonset $ \triangleq \{ x ~ \in ~ \wp a ~\mid~ (\exists t ~\in~ a)[x = \{t\}]\}$
				\item $\cap a \triangleq \{x ~\in~ \cup a ~\mid~ (\forall e ~\in~ a)[x ~\in~ e]\}$
				\end{itemize}

			\subsubsection{Infinity Axiom}
			\label{sec:infinity}
				$\exists I (\emptyset ~\in~ I ~\land~ \forall x (x ~\in~ I \Rightarrow \{x\} ~\in~ I ))$ ou\\
				$\exists I (\emptyset ~\in~ I ~\land~ \forall x (x ~\in~ I \Rightarrow x ~\cup~ \{x\} ~\in~ I ))$\\

				Esse axioma \'e garantido pois\\
				$\{x\} \ne x $\\
				$ x \cup \{x\} \ne x $

				Com ele, somos capazes de montar o seguinte conjunto infinito:\\
				I = $\{ \emptyset, \{\emptyset\}, \{\{\emptyset\}\}, \cdots \}$ ou\\


			\subsection{Rela\c{c}\~oes, Fun\c{c}\~oes e Fun\c{c}\~oes parciais na ZFC}
			\subsubsection{Definindo um par ordenado}
				A opera\c{c}\~ao de par do Kuratowski:\\
				$ (x, y) = \{ \{x\}, \{x, y\}\} $, como vimos anteriormente na se\c{c}\~ao~\ref{sec:separation}

			\subsubsection{Rela\c{c}\~oes}
				
				Def: Sejam A,B conjuntos, R \'e uma rela\c{c}\~ao \underline{entre} A e B se R $\subseteq$ A $\times$ B.

				Dessa forma, 
				\begin{itemize}
				\item f(a) = b $\rightsquigarrow$ (a,b) $\in$ f
				\end{itemize}

				\begin{itemize}
				\item[-] Uni\~ao Disjunta: $A \uplus B = (\{0,a\} \times A) \cup (\{1,b\} \times B)$				
				\end{itemize}

				Sendo R uma rela\c{c}\~ao sobre o conjunto $\mathbb{N}$ (R $\subseteq$ A $\times$ A), R pode ser:
				\begin{itemize}
				\item x R x : Reflexiva $\rightsquigarrow$ `` $ =, ~\leq, ~\geq, ~\subseteq $ ''
				\item x R y $\Rightarrow $ y R x  : Sim\'etrica $\rightsquigarrow$ `` $ = $  ''
				\item x R y $\land$ y R z $\Rightarrow$ x R z : Transitiva $\rightsquigarrow$  `` $ =, ~\leq, ~\geq, ~<, ~>, ~\subseteq $ ''
				\end{itemize}

				Ainda existem outras propriedas como essas, como a \underline{Antireflexiva} ou \underline{Antisem\'etrica}.

			\subsubsection{Rela\c{c}\~oes de Equival\^encia}
			  	Uma rela\c{c}\~ao sobre um conjunto A \'e chamada \textbf{rela\c{c}\~ao de equival\^encia} se ela for \underline{reflexiva},
			  	\underline{sim\'etrica} e \underline{transitiva}.

			  	O conjunto de todos os elementos que s\~ao relacionados a um elemento \textbf{a} de \textbf{A} \'e chamado de \underline{classe de equival\^encia} de \textbf {a}. Isso implica que:
			  	\begin{center}
			  	$\cup [a] = A$
			  	\end{center}

			  	\begin{itemize}
			  	\item $[a] \cap [b] = \emptyset$ quando [a] $\neq$[b]\\
			  	\end{itemize}

			  	Uma parti\c{c}\~ao de um conjunto S \'e uma cole\c{c}\~ao de subconjuntos disjuntos n\~ao vazios de S. A uni\~ao de todas as parti\c{c}\~oes resulta, portanto, em S. Em outras palavras, os subconjuntos $A_i$ formam parti\c{c}\~oes de S se e somente se\\
			  	$A_i \neq \emptyset$\\
			  	$A_i \cap A_j = \emptyset, quando~i \neq j$\\
			  	$\cap A_i = S$

				Podemos definir classes de equival\^encia como:\\
				\begin{center}
				$[x/\backsim] \triangleq \{a \in A \mid x \backsim a \}$\\
				$[A/\backsim] \triangleq \{c \in \wp(A) \mid \exists x ~ C = [x/\backsim] \}$
				\end{center}

				Seja x,y $\in$ A, e $\backsim$ uma rela\c{c}\~ao de equival\^encia no A:
				\begin{itemize}
				\item[-] $[x/\backsim] = [y/\backsim] \Leftrightarrow x \backsim y$
				\item[-] $[x/\backsim] = [y/\backsim] \Leftrightarrow [x/\backsim], se x \backsim y; \emptyset$, se n\~ao.
				\item[-] $\cup\{[x/ \backsim] \mid x \in a \} = A $
				\end{itemize}
		
			\subsubsection{Fun\c{c}\~oes}
		
		\subsection{Currying}
			Dada uma fun\c{c}\~ao f do tipo $f:(X \times X) \rightarrow Z$, ent\~ao a t\'ecnica de \textbf{currying} a torna $(f): X \rightarrow (Y \rightarrow Z)$.
			Isto \'e, currying torna um param\^etro do tipo X e retorna uma fun\c{c}\~ao do tipo $Y \rightarrow Z$.

			\begin{proof}
			Achar um $\phi ( (x,y) \rightarrow A \rightarrowtail\!\!\!\!\!\rightarrow (x \rightarrow (y \rightarrow A)))$\\
			$\phi(F) = G$, onde G \'e definida pela,\\
			$G(x)(y) = g$, ode g \'e definida pela,\\
			$g(y), f(x,y)$
			\end{proof} 
		
		\subsection{Cardinais}
			Seja A um conjunto. O que \'e |A|?
			\begin{itemize}
			\item[c1.] A $=_c$ |A|
			\item[c2.] A $=_c$ B $\Leftrightarrow$ |A| = |B|
			\item[c3.] Para todo conjunto de conjuntos $\epsilon$,
						$\{|x| \mid x \in \epsilon \}$ \'e conjunto.
			\end{itemize}

			Aqui estamos dando uma definini\c{c}\~ao fraca (\textbf{weak}) dos cardinais. N\~ao iremos provar essas tr\^es propriedas nesse momento, e vamos apenas t\^e-las como verdade.

		\subsection{Os Axiomas de Peano}
			Structed set: $(\mathbb{N}; 0; S)$, onde 0 $\in \mathbb{N}$ e S:$\mathbb{N} \rightarrow \mathbb{N}$\\

			\begin{enumerate}
			\item 0 $\in \mathbb{N}$
			\item S: $\mathbb{N} \rightarrow \mathbb{N}$
			\item S: $\mathbb{N} \rightarrowtail \mathbb{N}$
			\item $(\forall x \in \mathbb{N})[S_n \neq$ 0 ]
			\item $(\forall x \subseteq \mathbb{N})[ [0 \in X \land (\forall n \in \mathbb{N}) [n \in X \Rightarrow S_n \in X ]] \Rightarrow X = \mathbb{N}]$
			\end{enumerate}

			O axioma de peano 5 \'e o que nos permite realizar indu\c{c}\~ao matem\'atica. Observe:
			\begin{itemize}
			\item $\forall x \subseteq \mathbb{N}$ corresponde a \textbf{base}.
			\item $\forall n \in \mathbb{N}) [n \in X \Rightarrow S_n \in X $ corresponde ao \textbf{passo indutivo}.
			\item $n \in X$ corresponde a \textbf{hipot\'ese indutiva}.
			\end{itemize}
		
		\subsection{Teorema da Recurs\~ao}
		\label{sec:recursion}
			Sejam: $(\mathbb{N}$, 0, S) um sistema de naturais conjunto E.\\
			$a \in E$\\
			h:$E \rightarrow E$\\
			Ent\~ao existe f:$ \mathbb{N} \rightarrow E$\\
			tal que: f(0) = a e f($S_n$) = h(f(n)).

		\subsection{Os Naturais na ZFC}
			Para estabelecermos os Naturais na ZFC, temos que garantir duas coisas:
			\begin{itemize}
			\item Exist\^encia de $\mathbb{N}$
			\item Singularidade de $\mathbb{N}$
			\end{itemize}

			Para tal, iremos precisar do Teorema da Recurs\~ao~(\ref{sec:recursion}).

			\subsubsection{Exist\^encia de $\mathbb{N}$}
				Seja J = todos os conjuntos X tal que satisfaz o ZF7 (\ref{sec:infinity})\\
				$J = \{X \in \wp I \mid \emptyset \in X \land (\forall x \in X)[S \in X]\}$\\
				\begin{multicols}{2}
				Seja $\mathbb{N} \triangleq \cap J$\\ 
				Seja $0_1 = \emptyset$\\ 
				Seja $S_1 = \lambda x.\{x\}$ \footnote{Visiste \ref{sec:lambda} para $\lambda$-Calculus e entender melhor esse ponto}\\ 

				Seja $\mathbb{N} \triangleq \cap J$\\
				Seja $0_2 = \emptyset$\\
				Seja $S_2 = \lambda x.x \cup \{x\} $ \footnote{Veja nota 1} 
				\end{multicols}

			Encaixando com os Axiomas de Peano:
			\begin{enumerate}
			\item $(\forall x \in J)[\emptyset \in X]$, ent\~ao $\emptyset \in \cap J$ e $\emptyset \in \mathbb{N}$
			\item Tamb\'em, pela pr\'opria defini\c{c}\~ao de J
			\item $a \neq b \Leftrightarrow S_a \neq S_b$ ou $a \neq b \Leftrightarrow \{a\} \neq \{b\}$
			\item $\forall x \{x\} \neq \emptyset$
			\item Seja $X \subseteq \mathbb{N}$, tal que\\	
					0 $\in$ X\\
					$(\forall x \in X)[S_X \in X]$\\
					Seja $n \in \mathbb{N}$\\
					$\exists p : x = \{p\}$\\
					--> Mesmo que $\{p\} \in \cap J = \mathbb{N}$\\
					--> Mesmo que $n \in X$ e $x \geq \mathbb{N}$
			\end{enumerate}

			Basicamente, podemos descrever $\mathbb{N}$ de duas maneiras, agora:
			\begin{multicols}{2}
				\noindent 0 \hspace{4bp} $\emptyset$ = $\emptyset$\\
				1 \hspace{4bp} $\{\emptyset\} = \{0\}$\\
				2 \hspace{4bp} $\{\{\emptyset\}\} = \{1\}$\\
				3 \hspace{4bp} $\{\{\{\emptyset\}\}\} = \{2\}\\$
				$\vdots$\\
				\hspace{5bp} S = $\lambda$x.$\{x\}$ \\

				\noindent 0 \hspace{4bp} $\emptyset$ = $\emptyset$\\
				1 \hspace{4bp} $\{\emptyset\} = \{0\}$\\
				2 \hspace{4bp} $\{\emptyset, \{\emptyset\}\} = \{0,1\}$\\
				3 \hspace{4bp} $\{\emptyset, \{\emptyset\}\, \{\{\emptyset\}\}\}$\\
				$\vdots$\\
				\hspace{5bp} S = $\lambda x.x \cup \{x\} $
			\end{multicols}
			
			\subsubsection{Singularidade de $\mathbb{N}$}
				``$\mathbb{N}$ is unique up to isomorphism:''\\
				$\pi$ : ($\mathbb{N}_1$; $0_1$; $S_1$) $ \rightarrowtail\!\!\!\!\!\rightarrow $ ($\mathbb{N}_1$; $0_2$; $S_2$).
				\begin{itemize}
				\item[]\hspace{4bp}$\pi (0_1) = 0_2$
				\item[]\hspace{4bp}$\pi (S_1 n_1) = S_2 \pi (n_1)$
				\end{itemize}

				Se tra\c{c}armos um paralelo com o Teorema da Recurs\~ao (\ref{sec:recursion}), para tentearmos provar a singularidade de $\mathbb{N}$, podemos realizar as seguintes associa\c{c}\~oes:
				\begin{itemize}
				\item $\mathbb{N}$ : $\mathbb{N}_1$
				\item E: $\mathbb{N}_2$
				\item a: $0_2$
				\item h: $S_2$
				\end{itemize}

				\begin{proof}
					$\pi : \mathbb{N}_1 \Rightarrow \mathbb{N}_2$\\
					$\pi[\mathbb{N}_1] = \mathbb{N}_2$
						\begin{itemize}
						\item $0_2 \in \pi[\mathbb{N}_1]$\\
						--> Como $\pi(0_1) \Rightarrow S_2n_2 \in \pi[\mathbb{N}_1]$
						\item $n_2 \in \pi[\mathbb{N}_1]$\\
						--> Suponha que $n_2 \in \pi[\mathbb{N}_1]$ --> H.I\\
						--> $(\exists n_1 \in \mathbb{N}_1)[\pi(n_1) = n_2]$\\
						--> $\pi(S_1n_1) = S_2(\pi(n_2))$ que $ = S_2n_2$\\
						\end{itemize}
				
						MAIS COISA AQUI DEPOIS

				\end{proof}

		\subsection{String Recursion}
			\begin{itemize}
			\item Dado [ ] $\in$ [$\mathbb{N}$]
			\item Se n $\in \mathbb{N}$, e L $\in [\mathbb{N}$], ent\~ao (n:L) $\in$ [$\mathbb{N}$]
			\end{itemize}

			Exemplo: 2:3:4:[ ] = [2,3,4]\\

			Alguns exemplos de fun\c{c}\~oes recursivas que podemos definir utilizando String Recursion:

			Iszero --> O elemento \'e 0?
			\begin{itemize}
			\item iszero : $[\mathbb{N}] \rightarrow \mathbb{B}$
			\item iszero 0 = true
			\item iszero $S_n$ = false \\
			\end{itemize}

			Emtpy --> A lista est\'a vazia?
			\begin{itemize}
			\item empty : $[\mathbb{N}/ \rightarrow \mathbb{B}$
			\item empty [ ] = true
			\item empty (x:$x_s$) = false\\
			\end{itemize}

			Soma duas listas:
			\begin{itemize}
			\item ++ : $[\mathbb{N}] \rightarrow [\mathbb{N}] \rightarrow [\mathbb{N}] $
			\item '[ ] ++ $y_s$ = $y_s$
			\item (x:$x_s$) ++ $y_s$ = x:($x_s$ ++ $y_s$)\\
			\end{itemize}
			Ex: [1,2] ++ [6,7,8,9]\\
			= 1:2:[ ] ++ [6,7,8,9]\\
			= 1:(2:[ ] ++ [6,7,8,9])\\
			= 1:(2:([ ] ++ [6,7,8,9]))\\
			= 1:2:[6,7,8,9]\\
			= [1,2,6,7,8,9]\\

			Inverte a lista:
			\begin{itemize}
			\item reverse : $[\mathbb{N}] \rightarrow [\mathbb{N}] $
			\item reverse [ ] = [ ]
			\item reverse [x] = [x]
			\item revese (x:xs) = reverse xs ++ [x]\\
			\end{itemize}

			xs $\sqsubseteq$ ys se xs \'e um prefixo de ys
			\begin{itemize}
			\item $\sqsubseteq$ : $[\mathbb{N}] \rightarrow [\mathbb{N}] \rightarrow \mathbb{B}  $
			\item (x:xs) $\sqsubseteq$ [ ] = false
			\item '[ ] $\sqsubseteq$ ys = true
			\item (x:xs) $\sqsubseteq$ (y:ys) = (x = y) $\land$ xs $\sqsubseteq$ ys\\
			\end{itemize}
			Ex: [2,3,4,5] $\sqsubseteq$ [2,3,5,7]\\
			= (2 = 2) $\land$ ([3,4,5] $\sqsubseteq$ [3,5,7])\\
			= (3 = 3) $\land$ ([4,5] $\sqsubseteq$ [5,7])\\
			= (4 = 5) $\land$ ([5] $\sqsubseteq$ [7])\\
			= FALSE\\

			$\in$:
			\begin{itemize}
			\item $\in$ : $\mathbb{N} \rightarrow [\mathbb{N}] \rightarrow \mathbb{B}$
			\item n $\in$ x [ ] = false
			\item x $\in$ (x:xs) = (n = x) $\vee$ (n $\in$ xs) \\
			\end{itemize}

			Find:
			\begin{itemize}
			\item find : $\mathbb{N} \rightarrow [\mathbb{N}] \rightarrow \mathbb{B}$
			\item find n [ ] = 0
			\item find n (n:nx) = 0
			\item find n (x:xs) = 1 + find n xs\\
			\end{itemize}

			Sum:
			\begin{itemize}
			\item sum : $[\mathbb{N}] \rightarrow \mathbb{N}$
			\item sum [ ] = 0
			\item sum (x:xs) = x + sum xs\\
			\end{itemize}

			$\oplus$
			\begin{itemize}
			\item $\oplus$ : $[\mathbb{N}] \rightarrow [\mathbb{N}] \rightarrow [\mathbb{N}]$
			\item '[ ] $\oplus$ ys = ys
			\item xs $\oplus$ [ ] = xs
			\item (x:xs) $\oplus$ (y:ys) = (x+y):(xs $\oplus$ ys)\\
			\end{itemize}

			Circle:
			\begin{itemize}
			\item circle :$(\mathbb{N} \rightarrow \mathbb{N} \rightarrow \mathbb{N}) \rightarrow [\mathbb{N}] \rightarrow [\mathbb{N}] \rightarrow [\mathbb{N}]$
			\item circle f [ ] ys = [ ]
			\item f xs [ ] = [ ]
			\item f (x:xs) (y:ys) = [f x y]:(circle f xs ys)\\
			\end{itemize}

	\section{$\lambda$-Calculus}
	\label{sec:lambda}
		\subsection{O conjunto de $\lambda$-termos}
		\subsection{$\alpha,~ \beta~ e~ \eta~$ conversions}
		\subsection{Booleanos naturais no $\Lambda$}
		\subsection{Combinators I, K, B, S}


\end{document}
